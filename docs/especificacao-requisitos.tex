\documentclass{article}
\usepackage{graphicx}
\usepackage[utf8]{inputenc}
\usepackage{hyperref}
\usepackage[brazil]{babel}

\title{Especificação de Requisitos: Jogo Chaturanga}
\author{André Fernandes, Eduardo Petry, Gabriel Holstrein}
\date{May 2022}

\hypersetup{
    colorlinks=true,
    linkcolor=blue,
    filecolor=magenta,      
    urlcolor=cyan
    }
    
\begin{document}

\begin{titlepage}
\begin{center}
    {\large Universidade Federal de Santa Catarina - UFSC}\\[0.2cm]
    {\large Departamento de Informática e Estatística - INE}\\[0.2cm]
    {\large Curso de Ciências da Computação}\\[0.2cm]
    {\large INE5417 - Engenharia de Software I}\\[5.1cm]
    {\bf \huge Especificação de Requisitos: \\ Jogo Chaturanga}\\[5.1cm]
\end{center}

\begin{flushright}
    {\bf \large  Alunos:} \\
        André Filipe da Silva Fernandes \\
        Eduardo Vicente Petry \\
        Gabriel Holstein Meireles \\
    [0.7cm]
        
    {\bf \large Professor:}\\
        Ricardo Pereira e Silva \\
    [0.7cm]
\end{flushright}

\begin{center}
    {\large Florianópolis}\\[0.2cm]
    {\large 2022}
\end{center}

\end{titlepage} %término da "capa"
\pagebreak

\section{Introdução}

Versão 0.0
09/05/2022

\begin{center}
    \begin{tabular}{|c|p{5cm}|c|p{5cm}|}
    
    \hline
    Versão & Autores & Data & Ação \\

    \hline 
    0.0 &
    André Filipe da Silva Fernandes, Eduardo Vicente Petry, Gabriel Holstein Meireles &
    09/05/2022 &
    Estabelecimento dos requisitos \\
    
    \hline
    
    \end{tabular}
\end{center}

\subsection{Objetivo do Desenvolvimento}
    Desenvolvimento de um software para computador que permita a disputa local entre dois jogadores em uma partida de Chaturanga, um antigo jogo indiano.

\subsection{Referências}
    \url{https://www.chessvariants.com/historic.dir/chaturanga.html} \\
    \url{https://en.wikipedia.org/wiki/Chaturanga} \\
    
\section{Visão Geral do Sistema}
\subsection{Arquitetura do Programa}
    Programa orientado a objetos escrito na linguagem Python.
    
\subsection{Descrição do Jogo}
    Chaturanga é um jogo composto por dois jogadores, um tabuleiro 8x8 e 16 peças para cada jogador.
    As 16 peças compõem um exército e se dividem em oito Padati, dois Ratha, dois Ashwa, dois Gaja, um Mitri e um Raja.
    
    A distribuição do exército no tabuleiro para o início da partida se dá como na imagem \ref{figura:initial_board}.
    
    \begin{figure}[h]
    \centering
    \includegraphics[width=7cm]{imgs/initial_board.png}
    \caption{Configuração Inicial do Tabuleiro}
    \label{figura:initial_board}
    \end{figure}

    O jogador com as peças brancas deve iniciar a partida e, então, os turnos alternam entre os dois jogadores. O objetivo do jogo é capturar o Raja inimigo ou aniquilar o exército adversário.

    A cada turno, o jogador escolhe uma peça de sua cor que está no tabuleiro e a move para alguma casa vazia ou captura uma peça adversária.
    
    As regras de movimento para cada peça são as seguintes:

    \begin{itemize}
        \item Padati (Soldado): anda uma casa diretamente para frente ou, no caso de existir uma peça inimiga na sua diagonal direta, captura uma peça inimiga em sua diagonal direta, como na figura \ref{figura:padati_moves}.
        
        \item Ratha (Carruagem): anda em movimento ortogonal ao tabuleiro, ou seja, tanto para frente/trás quanto para esquerda/direita e se move quantas casas forem desejadas, como na figura \ref{figura:ratha_moves}.
        
        \item Ashwa (Cavalo): anda imediatamente na diagonal oposta de um retângulo 3x2 e ignora qualquer peça que exista no caminho, como na figura \ref{figura:ashwa_moves}.
        
        \item Gaja (Elefante): anda duas casas em uma diagonal adjacente e ignora qualquer peça que exista no caminho, ou uma casa diretamente para frente, como na figura \ref{figura:gaja_moves}.
        
        \item Mitri (Ministro): anda uma casa em uma diagonal adjacente, como na figura \ref{figura:mitri_moves}.
        
        \item Raja (Rei): anda uma casa em qualquer direção adjacente, como na figura \ref{figura:raja_moves}.
        
    \end{itemize}
    
    Nota-se que no caso das peças que tem movimento de várias casas, elas param ao colidir com uma peça aliada ou ao capturar uma inimiga a não ser que seja especificado que ignoram outras peças. Em caso de movimento que resulta em captura, a peça que fez o movimento passa a ocupar a casa previamente pertencente a capturada.

    No caso especificamente do Padati, há a possibilidade de promoção para virar um Mitri. Para isso, é necessário que a peça encoste na borda do tabuleiro pertencente ao exército do adversário.
    
    \begin{figure}[h]
    \centering
    \includegraphics[width=5cm]{imgs/padati_moves.png}
    \caption{Movimentação do Padati (Soldado)}
    \label{figura:padati_moves}
    \end{figure}
    
    \begin{figure}[h]
    \centering
    \includegraphics[width=5cm]{imgs/ratha_moves.png}
    \caption{Movimentação do Ratha (Carruagem)}
    \label{figura:ratha_moves}
    \end{figure}

    \begin{figure}[h]
    \centering
    \includegraphics[width=5cm]{imgs/ashwa_moves.png}
    \caption{Movimentação do Ashwa (Cavalo)}
    \label{figura:ashwa_moves}
    \end{figure}
    
    \begin{figure}[h]
    \centering
    \includegraphics[width=5cm]{imgs/gaja_moves.png}
    \caption{Movimentação do Gaja (Elefante)}
    \label{figura:gaja_moves}
    \end{figure}
    
    \begin{figure}[h]
    \centering
    \includegraphics[width=5cm]{imgs/mitri_moves.png}
    \caption{Movimentação do Mitri (Ministri)}
    \label{figura:mitri_moves}
    \end{figure}

    \begin{figure}[h]
    \centering
    \includegraphics[width=5cm]{imgs/padati_moves.png}
    \caption{Movimentação do Raja (Rei)}
    \label{figura:raja_moves}
    \end{figure}

\subsection{Premissas de Desenvolvimento}
    \begin{itemize}
        \item O programa será desenvolvido com a linguagem de programação Python 3 sob o paradigma de Orientação a Objetos.
        
        \item O programa deverá apresentar uma interface bidimensional, onde cada jogador irá interagir com o uso do mouse.
        
        \item O jogo será implementado em português.
        
        \item Uma partida será iniciada junto com a execução do programa.
        
    \end{itemize}
    
\section{Requisitos de Software}

\subsection{Requisitos Funcionais}
    \begin{itemize}
        %\item Requisito Funcional 0 - Iniciar partida (?) precisamos decidir se começa com o tabuleiro configurado ou se o botão de reiniciar partida é inicialmente iniciar e depois troca de nome
    
        \item Requisito Funcional 1 - Reiniciar partida: no turno do jogador, ele poderá clicar em um botão escrito "Reiniciar partida" para reiniciar o jogo. Caso o botão seja clicado, será considerado vitória do jogador adversário e o tabuleiro retornará para a configuração inicial.
        
        \item Requisito Funcional 2 - Escolher peça: no turno do jogador, ele poderá clicar em uma peça qualquer que está no tabuleiro. Ao clicar na peça, as possíveis casas para qual tal peça pode se mover devem tornar-se amarelas. Deve ser possível que o jogador escolha outra peça antes de fazer o movimento, basta ele clicar em outra peça.
        Nota-se que cada peça possui movimentos distintos dependendo do seu tipo e isso deve ser obedecido na hora de determinar a posição das casas amarelas.
        
        \item Requisito Funcional 3 - Mover peça: no turno do jogador, após escolher uma peça, o jogador poderá movê-la para uma das casas amarelas ao clicar nela. Sempre que uma peça é movida, deve-se verificar se a condição de vitória foi atingida conforme a Descrição do jogo e, se sim, a partida é finalizada e o jogador da vez é o vencedor. Se não, o turno é finalizado e torna-se a vez do jogador adversário.
        
        %\item Requisito Funcional 2 - Mover peça: no turno do jogador, ele deverá clicar em uma peça de seu controle que ainda está no tabuleiro para realizar um movimento. A funcionalidade deve obedecer os seguintes requisitos:
        %\begin{itemize}
            %\item Cada peça possui regras de movimento distintas e devem seguir as regras especificadas na Descrição do jogo;
            %\item Ao clicar numa peça, as casas para qual ela pode se mover tornam-se amarelas;
           %\item Caso seja escolhido uma peça qualquer e selecionado uma casa para qual ela não pode se mover, deve-se informar jogada ilegal e o jogador deve novamente clicar em uma peça para poder realizar o movimento.
        %\end{itemize}
        %Após um movimento legal de peça, deve-se verificar se a condição de vitória foi atingida conforme especificado na Descrição do jogo. Se a partida não finalizou, o turno é encerrado e começa a vez do jogador adversário.
    \end{itemize}

\subsection{Requisitos Não Funcionais}
    \begin{itemize}
        \item Requisito Não Funcional 1 - Especificação de Projeto: 
    Além do código em Python deve ser produzida especificação de projetos baseada em UML2, através da ferramenta Visual Paradigm.
    
        \item Requisito Não Funcional 2 - Interface Gráfica:
    O programa deve implementar a interface gráfica utilizando as bibliotecas Tkinter ou Pyqt5.
    \end{itemize}

\subsection{Esboço da Interface}

    \begin{figure}[h]
    \centering
    \includegraphics[width=15cm]{imgs/interface.jpg}
    \caption{Esboço da Interface}
    \label{figura:interface}
    \end{figure}
    

    
\end{document}
